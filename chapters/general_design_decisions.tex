\chapter{General Design Decisions}\label{chapter:general_design_decisions}

\section{Architectural Overview}

\section{The Framework}
The framework comprises of an Isolate System, a Registry, a Message Queuing System and a Message Broker System.
  \subsection{IsolateSystem}
  An Isolate System is analogous to an actor system. Just as actor system is comprised of a group of actors working together, a group of related isolates form an isolate system. A ‘Bootstrapper’ in physical node can bootstrap several Isolate Systems. Nevertheless, a logical Isolate System is not limited to a single physical node. The isolates spawned by an Isolate System can be distributed across several remote systems.
  When an Isolate System is bootstrapped, it spawn controller

  \subsection{Controller}
  A controller creates a local router and instructs the router to create a certain number of local or remote workers with a certain routing policy. Whenever a message arrives in the controller, it delegates the message to the proper router, which then delegates the message to an isolate using the specified routing policy.
  \subsection{Router}
  A router is spawned by the controller. It spawns and is responsible for a group of identical workers. Since an isolate is single threaded, multiple instances of an isolate can be created by a router for concurrency. When a message arrives to a router from a controller, the router, based on its defined routing policy, delegates the message to one of the worker isolates.
  For proper load balancing among a group of isolates, a router uses a routing policy. The routing policies that are available in the framework are:

  \subsubsection{Round Robin Router}
    Messages are passed in round-robin fashion to child isolates.

  \subsubsection{Random Router}
  Randomly picks an isolate and the message is passed to that isolate.

  \subsubsection{Broadcast Router}
  Replicated and sends message to all child isolates.

  \subsection{Worker}
  Worker isolates carry out the business logic tasks. However, if a certain task is too complex, the isolate can divide the tasks into subtasks and spawn temporary isolates to carry out those subtasks. The temporary isolates, can again spawn other child temporary-isolates to further divide the subtasks into sub-subtask. The temporary isolates terminate once the subtask has been carried out.
  \subsubsection{Specialized Worker: Proxy}
  \subsubsection{Specialized Worker: FileMonitor}

\section{The Registry}
The Isolate Registry is a central registry system where meta-data of isolates and isolate systems are stored. It bootstraps all isolates systems and stores the current deployed location of every isolate system. It’s functions are:
  \subsection{RESTful API of Registry}
  The registry provides a REST API to perform the operations on the connected nodes. One can query the ‘Registry’ using the ‘GET’ method of REST to fetch the list of the nodes that are connected to the Registry.
  \begin{itemize}
  \item GET list of connected nodes
  \item GET details of the running isolate systems on the node
  \item POST command to shutdown an isolate system
  \item POST command to kill a particular isolate pool
  \end{itemize}

The functions of Registry are:
\begin{itemize}
  \item Bootstrap an isolate system, during runtime, in local or remote virtual machine
  \item Provide a way to deploy, update or remove an ‘isolate system’
  \item Return information about the isolate systems running isolates by querying the individual isolate systems of a node
\end{itemize}
The registry does not need to persist any data as all the information about isolate systems are queried and generated “on the fly”.

  \subsection{Management Interface}
  The deployment of isolates can be managed through the web interface provided by Isolate Registry

\section{Message Queuing System (MQS)}
Since, the basis of this system is going to be messages, a message queue system is an important part of this framework. It is an isolate that fetches messages from message broker system and dispatches to respective the isolate system, of connected node, where the isolate belongs. Whenever a new isolate system starts up, the isolate system connects with the message queuing system.
It consists of two major components: Enqueuer \textendash which enqueues messages and Dequeuer \textendash which dequeues messages
  \subsection{Enqueuer}
  Enqueuer is a single separate isolate. A Message Queuing System has only one enqueuer, which basically receives message from the MQS and sends message to message broker system \textendash RabbitMQ~\ref{sec:rabbitmq} via STOMP~\ref{sec:stomp} protocol.

  \subsection{Dequeuer}
  As opposed to Enqueuer, a Message Queuing System maintains several Dequeuers.

\section{Message Broker System \textendash RabbitMQ}

\section{Activators}
\subsection{SystemBootstrapper}
\subsection{IsolateDeployer}

\section{Hot Deployment of Isolates and Isolate Systems}

\section{Migration of Isolates}

\section{Remote Isolates}

\section{Typical Message Flow in the System}

\section{Key Concepts}

\begin{itemize}
  \item The framework does not guarantee the delivery of a message.
  \item All the messages send by the framework is based on ‘fire and forget’concept.
  \item A message is delivered at most once.

\end{itemize}
