\chapter{Introduction}\label{chapter:introduction}

\section{Section}
Citation test~\parencite{latex}.
The purpose of this thesis is to build a framework based on the actor programming model and the Dart language. The framework unifies applications across devices, client and server and also supports migration of actors in a distributed system. So, first of all we should briefly overview the actor programming model and how it can be realized efficiently in the Dart language. Although the actor model was introduced in mid 1980s and there had been programming languages like Erlang that implemented it, only now it has started gaining wide popularity in distributed systems. Especially after the introduction of Scala and Akka, the actor model has been gaining good popularity. The Dart programming language provides a homogeneous system that encompasses both client as well as server as the Dart Virtual Machine runs in servers as well as in browsers. This particular nature of the Dart language makes it possible to create a fully distributed application in which isolates (the actor like entities of Dart language) may run everywhere \textemdash{} in servers, in desktop browsers and even in mobile browsers.

\subsection{Subsection}
See~\autoref{fig:sample}.

\begin{figure}[htsb]
  \centering
  \includegraphics{logos/tum}
  \caption[Example figure]{An example for a figure.}\label{fig:sample}
\end{figure}

\section{Section}

See~\autoref{tab:sample}, \autoref{fig:sample-drawing}, \autoref{fig:sample-plot}, \autoref{fig:sample-listing}.

\begin{table}[htsb]
  \caption[Example table]{An example for a simple table.}\label{tab:sample}
  \centering
  \begin{tabular}{l l l l}
    \toprule
      A & B & C & D \\
    \midrule
      1 & 2 & 1 & 2 \\
      2 & 3 & 2 & 3 \\
    \bottomrule
  \end{tabular}
\end{table}

\begin{figure}[htsb]
  \centering
  % This should probably go into a file in figures/
  \begin{tikzpicture}[node distance=3cm]
    \node (R0) {$R_1$};
    \node (R1) [right of=R0] {$R_2$};
    \node (R2) [below of=R1] {$R_4$};
    \node (R3) [below of=R0] {$R_3$};
    \node (R4) [right of=R1] {$R_5$};

    \path[every node]
      (R0) edge (R1)
      (R0) edge (R3)
      (R3) edge (R2)
      (R2) edge (R1)
      (R1) edge (R4);
  \end{tikzpicture}
  \caption[Example drawing]{An example for a simple drawing.}\label{fig:sample-drawing}
\end{figure}

\begin{figure}[htsb]
  \centering

  \pgfplotstableset{col sep=&, row sep=\\}
  % This should probably go into a file in data/
  \pgfplotstableread{
    a & b    \\
    1 & 1000 \\
    2 & 1500 \\
    3 & 1600 \\
  }\exampleA
  \pgfplotstableread{
    a & b    \\
    1 & 1200 \\
    2 & 800 \\
    3 & 1400 \\
  }\exampleB
  % This should probably go into a file in figures/
  \begin{tikzpicture}
    \begin{axis}[
        ymin=0,
        legend style={legend pos=south east},
        grid,
        thick,
        ylabel=Y,
        xlabel=X
      ]
      \addplot table[x=a, y=b]{\exampleA};
      \addlegendentry{Example A};
      \addplot table[x=a, y=b]{\exampleB};
      \addlegendentry{Example B};
    \end{axis}
  \end{tikzpicture}
  \caption[Example plot]{An example for a simple plot.}\label{fig:sample-plot}
\end{figure}

\begin{figure}[htsb]
  \centering
  \begin{tabular}{c}
  \begin{lstlisting}[language=SQL]
    SELECT * FROM tbl WHERE tbl.str = "str"
  \end{lstlisting}
  \end{tabular}
  \caption[Example listing]{An example for a source code listing.}\label{fig:sample-listing}
\end{figure}
